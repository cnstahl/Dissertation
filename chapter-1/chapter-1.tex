% !TEX root = ../thesis.tex

\chapter{Introduction}
\label{chp:intro}

\begin{quotation}
	 ``I should see the garden far better," said Alice to herself, ``if I could get to the top of that hill: and here’s a path that leads straight to it—at least, no, it doesn’t do that—" (after going a few yards along the path, and turning several sharp corners), ``but I suppose it will at last. But how curiously it twists! It's more like a corkscrew than a path! Well, this turn goes to the hill, I suppose—no, it doesn't! This goes straight back to the house! Well then, I’ll try it the other way."
	And so she did: wandering up and down, and trying turn after turn, but always coming back to the house, do what she would.~[\hyperlink{cite.\therefsection @Carroll2002Alice}{Carroll}]
\end{quotation}

The dynamics of any quantum system has steady states, which are states that the dynamics reach at late times regardless of the precise details of the initial conditions. If the dynamics are completely generic, there can only be a single steady state. Completely generic dynamics must not treat any particular pure state in Hilbert space differently than any other, so the steady state must be the maximally mixed state,
\begin{align}
\rho = \mathds{1} = \frac{1}{|\mathcal{H}|} \sum_{i=1}^N \ket{\psi_i} \bra{\psi_i}.
\end{align}
As a result, any dynamics that have different steady states must be nongeneric in some way. 

When quantum dynamics have multiple steady states, they can become useful for information storage. If the system starts in one steady state or another, it will remain there, so information can be stored in this choice. Thus, the problems of finding quantum dynamics with nontrivial steady states, or especially multiple steady states, are central problems in the theory of condensed matter systems.

\section{Quantum dynamics and symmetry}

The simplest and most common way to arrive at nontrivial steady states is to impose a symmetry on the dynamics. This symmetry, which might be discrete or continuous, breaks Hilbert space into symmetry sectors labeled by quantum numbers that correspond to the symmetry. An illuminating example of a discrete symmetry is the Ising symmetry, $S = \prod_i X_i$, which flips all the spins~\cite{Ising1925Contribution}. This is a $\mathbb{Z}_2$ symmetry, and the two resulting symmetry sectors are symmetric sector $\mathcal{H}_\text{sym}$ and antisymmetric sector $\mathcal{H}_\text{anti}$. A state $\ket{\psi}$ is in the symmetric sector if $S\ket{\psi} = \ket{\psi}$, or in the antisymmetric sector if $S\ket{\psi} = -\ket{\psi}$. Any state can be written as $\ket{\psi} = \ket{\psi_\text{s}} + \ket{\psi_\text{a}}$, with $\ket{\psi_\text{s}}$ a symmetric state and $\ket{\psi_\text{a}}$ an antisymmetric state. The symmetry enforces that symmetric states only evolve into symmetric states, and antisymmetric states only evolve into antisymmetric states.
Generic dynamics that obey the Ising symmetry have two steady states, the symmetric state 
\begin{equation}
\rho_\text{sym} = \frac{1}{|\mathcal{H}_\text{sym}|} \sum_{i \text{ sym}} \ket{\psi_i} \bra{\psi_i},
\end{equation}
and the antisymmetric state
\begin{equation}
\rho_\text{anti} = \frac{1}{|\mathcal{H}_\text{anti}|} \sum_{i \text{ anti}} \ket{\psi_i} \bra{\psi_i},
\end{equation}
which are the maximally mixed states on the symmetric Hilbert space and the antisymmetric Hilbert space, respectively.

Systems with continuous symmetries have conserved quantities, as described by Noether's theorem~\cite{Noether1918Theorem}. The symmetry sectors of such systems are labeled by the number operator $N$, which counts the conserved quantity associated with the symmetry. A particular symmetry sector consists of all states that share the same quantum numbers, and each symmetry sector possesses a single steady state that is the maximally mixed state within that sector.

As an example, any system with a time-invariant Hamiltonian $H$ possesses time-translation symmetry, whose associated conserved quantity is the energy $E$ measured by the Hamiltonian, which acts as the number operator. In this setting, an important sector is the ground state sector, which consists only of the ground state. Thus, the ground state is a steady state of energy-conserving dynamics. This steady state is a pure state, because there is only a single ground state.

Multiple symmetries in a system compound upon each other. A time-independent Hamiltonian with an Ising symmetry will have two ground states, one for each symmetry sector. Each of these states will be steady states of the dynamics. Any symmetry sector of such a system will be labeled by symmetry numbers for all of the symmetries.

So far, we have been discussing systems with symmetries imposed directly on the system. An alternative is to have the symmetry act jointly on the system and a bath, so that the system and bath are able to exchange the conserved quantity. This is especially common for energy conservation, where exchange of energy with the bath is controlled by the inverse temperature $\beta$. This allows dynamics to mix all of the states in the full Hilbert space, but with different coefficients, so that the steady state is
\begin{equation}
\rho_\text{thermal} = \frac{1}{Z} \sum_i e^{-\beta E_i} \ket{\psi_i} \bra{\psi_i} = \frac{1}{Z} \sum_i e^{-\beta H} \ket{\psi_i} \bra{\psi_i},
\end{equation}
where $H\ket{\psi_i} = E_i \ket{\psi_i}$ and $Z = \text{tr}(\exp -\beta H)$. Note that at $\beta=\infty$ this state reduces to just the ground state, so that ground state dynamics corresponds to zero temperature dynamics.
This steady state is called the canonical ensemble, whereas the previous steady state is called the microcanonical ensemble. When a system is allowed to exchange some other conserved quantity with a bath, the resulting steady state is called the grand canonical ensemble. 

While the examples we have discussed have had multiple steady states, they are not useful for information storage because of the practical problem of determining which symmetry sector the system is in. For the simplest example of the Ising symmetry, the problem of determining the symmetry sector reduces to measuring the horribly nonlocal Ising operator $S = \prod_I X_i$. Fortunately, an alternative exists, called spontaneous symmetry breaking. Consider the example of the time-independent Hamiltonian with Ising symmetry, commonly called the Ising model. In one space dimension (1d), there is a range of Hamiltonian parameters where the ground states of the two symmetry sectors are degenerate in the thermodynamic limit, with an energy gap exponentially small in system size $L$. Within this regime the symmetric ground state is $\ket{\bar{0}}+\ket{\bar{1}}$ and the antisymmetric ground state is $\ket{\bar{0}}-\ket{\bar{1}}$, where $\ket{\bar{0}}$ is the all-up state and $\ket{\bar{1}}$ is the all-down state. The degeneracy of these two states means that the individual states $\ket{\bar{0}}$ and $\ket{\bar{1}}$ also become ground states as $L\to \infty$, even though they do not belong to either symmetry sector. Furthermore, the mixing time between $\ket{\bar{0}}$ and $\ket{\bar{1}}$ vanishes in the thermodynamic limit. In the 1d Ising model this behavior only happens in the microcanonical ensemble in the ground state, but in higher dimensions it happens for a range of energies, and even happens in the canonical ensemble for a range of temperatures. 

Spontaneous symmetry breaking thus gives us a new type of nontrivial steady state, one that is not labeled by symmetry numbers but rather by a pattern of spontaneous symmetry breaking. Furthermore, the two symmetry-broken steady states are locally distinguishable by a single-site $Z$ operator, making them useful for information storage. In fact, the Ising model is a simple model for ferromagnetism, the behavior underlying many forms of classical information storage, such as magnetic tapes and hard drives~\cite{Brown2016Finite}. When the system can act as a memory at nonzero temperature we call it self-correcting.
The onset of spontaneous symmetry breaking is a phase transition in the dynamics. This phase transition also corresponds to a phase transition in the partition function, but for the present discussion that is beside the point.

The importance of spontaneous symmetry breaking is most familiar from the Landau classification of phases, in which all phases of matter are labeled by their pattern of spontaneous symmetry breaking~\cite{Landau1980Statistical}. This means that two systems are in the same phase if and only if they have the same pattern of spontaneous symmetry breaking. For example, any system with Ising symmetry can have two phases, one in which the symmetry is spontaneously broken and one in which it is not.

The Landau classification can also be weaponized to determine when different symmetry-breaking patterns are possible. The various Mermin-Wagner theorems tell use that discrete symmetries can only be broken when $d\ge 1$ at zero temperature and $d\ge 2$ at nonzero temperature, and that continuous symmetries can only be broken when $d \ge 2$ at zero temperature and $d \ge 3$ at nonzero temperature. This in turn tells us that if we are relying on spontaneous symmetry breaking for a classical memory, classical memories can only exist at nonzero temperature when $d \ge 2$.

To recap, the systems we have described all have steady states that are completely described by
\begin{itemize}
\item A symmetry
\item An ensemble, which may be either microcanonical or canonical with respect to each symmetry
\item A symmetry breaking pattern.
\end{itemize}
We will refer to such systems as ergodic or thermal, and to the process of reaching such steady states as ergodicity or thermalization.
The Landau paradigm, translated into the language of steady states, claims that all quantum dynamics are ergodic in this sense.
Whereas the Landau paradigm is usually phrased as a description of phases of matter, here we are instead interpreting it as a description of phases of dynamics. 

\section{Topological order and higher-form symmetry}

Landau's paradigm is certainly powerful, and for a time people believed that any nontrivial steady states beyond enforced symmetries came from spontaneous symmetry breaking. This belief began to crumble, however, with the discoveries of the integer quantum Hall effect~\cite{Klitzing1980Quantized} and the fractional quantum Hall effect~\cite{Tsui1982Extreme}.  These new phases of matter had no local order parameters and could not be described by ordinary spontaneous symmetry breaking. Instead, they motivated a new type of order, called topological order~\cite{Wen1990Topological}.

The toric code provides a very simple theoretical model of a topologically ordered system~\cite{Kitaev2003Fault}. The toric code generalizes earlier models of $\mathbb{Z}_2$ gauge theory~\cite{Wegner1971Duality} to create a quantum Hamiltonian with a nontrivial ground state degeneracy that depends on the topology of the system. To further emphasize the novelty of this model and its independence of the earlier symmetry-based approach, the ground state degeneracy does not rely on any symmetry and is robust to arbitrary perturbations. This seems new! Indeed, for a long time topological order was thought to be a source of nontrivial steady states beyond the Landau paradigm.

The other new feature of the toric code is that its steady states provide a quantum memory, instead of just a classical one. This is because the degeneracy of its symmetry-breaking ground states is robust to arbitrary perturbations, rather than just symmetry-preserving ones. In other words, in addition to the ground states $\ket{\bar{0}}$ and $\ket{\bar{1}}$, there are also ground states $\ket{\bar{+}} = \ket{\bar{0}} + \ket{\bar{1}}$ and $\ket{\bar{-}} = \ket{\bar{0}} - \ket{\bar{1}}$. While this superficially looks like the discussion of the Ising model, the difference is that a single $Z$ operator is sufficient to transition from the symmetric ground state of the Ising model to its antisymmetric ground state. Instead, for the toric code, any logical operator (operator that is able to transition the system between ground states) is stringlike and must act on at least $L$ sites for linear system size $L$.

While the 2d toric code only provides a quantum memory at zero temperature, the 4d toric code~\cite{Dennis2002Topological} does so even at nonzero temperatures. This means that there is a critical temperature or energy density below which there is a logical qubit's worth of steady states. Said another way, the 4d toric code is self-correcting. The question of the possibility of self-correction in 3d remains open.

It turns out that it is possible to reinterpret topological order as living comfortably within the Landau paradigm by considering higher-form symmetries~\cite{Nussinov2009Symmetry, Gaiotto2015Generalized, Lake2018Higher, McGreevy2022Generalized, Cordova2022Generalized}. In the language of higher-form symmetries, ordinary symmetries are 0-form symmetries and any $p$-form symmetry acts on codimension-$p$ submanifolds of space. The toric code has two 1-form symmetries, each of which consists of stringlike symmetry operators. Furthermore, $p$-form symmetries have $p$-dimensional order parameters, so the 2d toric code has stringlike order parameters. It turns out that the symmetry operators and the order parameters of the toric code are just its logical operators. In this language, topological order is just spontaneous symmetry breaking of higher-form symmetries.


But wait a minute! We said that the toric code remains a good quantum memory even in the presence of arbitrary perturbations, without any symmetry constraints. How, then, can topological order correspond to spontaneous breaking of higher-form symmetries? The resolution is that, in topologically ordered phases, the higher-form symmetries are emergent~\cite{Wen2019Higher}. This means that, even if they are explicitly broken at some microscopic length scale, the symmetries are present again at some longer length scale and may be spontaneously broken.

Even more recently, systems have been discovered that share some similarities with topological order, but have a richer set of steady states. These are the fracton models~\cite{Chamon2005Glassiness, Bravyi2011Order, Haah2011Code, Vijay2016Fracton,  NandkishoreHermele2019,Pretko2020Fracton}, which all have a number of ground states that diverges with system size. Their ground-state degeneracy, like that of topologically ordered systems, is robust to any small perturbation. As information-storage models, they are able to store a number of qubits proportional to their linear system size.

While there was some hope that fracton models would realize finite-temperature quantum memories, this has proved to not be the case~\cite{PremHaahNandkishore2017, Siva2017Marginally}. In fact, fracton models do not even provide classical memories at nonzero temperatures.

\section{Passive and active dynamics}

Given the previous discussion, it might seem like quantum memories at finite temperature in 3d are a lost hope. However, there is one way to avoid this conclusion, based on error correction. In this paradigm, the quantum system of interest does not just evolve blindly on its own, but rather some experimentalist steers the evolution by measuring and applying feedback. We will refer to these types of dynamics as active dynamics. 

Of course, we should put some restrictions on our active dynamics. First of all, the experimentalist should only be allowed to measure operators that are spatially local. Furthermore, we should require that the experimentalist is only allowed to perform classical operations on the measurement outcomes within some finite time window. This assumption is violated in the usual method of 2d toric code active error correction~\cite{Dennis2002Topological}, but error-correction schemes that do obey this constraint are called single-shot. Finally, we might want to only allow the experimentalist to perform classical computations on a spatially local set of measurement outcomes. We will call dynamics that obey this constraint local active dynamics, but we will mostly be interested in nonlocal active dynamics.

The surprising result of Ref.~\cite{Bombin2015SingleShot} is that single-shot quantum error correction is possible in 3d. Unfortunately, that reference uses a complicated error-correcting code called the gauge color code~\cite{Bombin2015Gauge}. Despite this important result, the gauge color code and the possibility of single-shot quantum error correction remain poorly understood. 

Interesting connections exist between single-shot quantum error correction and higher-form symmetries~\cite{Roberts2017SPTO, Kubica2018Ungauging}. In particular, Ref.~\cite{RobertsBartlett2020} showed that it is possible to use the gauge color code to build a symmetry-protected self-correcting quantum memory. This means that the memory is self correcting as long as a particular (higher-form) symmetry is enforced.

\section{Hilbert-space shattering and multipole symmetry}

With topological order tamed back into the Landau paradigm, it is once again tempting to believe that all quantum dynamics are thermal on long enough timescales. However, there are a few remaining exceptions. Integrable systems possess a macroscopic number of symmetries so that the number of symmetry sectors grows exponentially with system volume. However, generic perturbations will pick out a finite number of steady states, so we will leave these systems aside.

Many-body localization (MBL) is a fascinating phenomenon wherein strong disorder can trap a many-body system in steady states that are not labeled by obvious symmetry numbers. There have been intriguing arguments for the existence of MBL, and even proofs of the stability of MBL in 1d. However, these proofs have been called into question. Physically, MBL systems may suffer from avalanche instabilities that originate in rare regions where the disorder is not strong enough and travel across the system, thermalizing it. Thus, it remains an open question whether MBL results in infinite-lifetime steady states or merely very long-lived states, so we will not address the phenomenon further in this thesis. 

More recently, systems have been discovered that exhibit Hilbert-space shattering or fragmentation, in which the dynamics are nonergodic even without disorder. Shattering results from an interplay of symmetry and locality. The earliest examples rely on dipole symmetries, which conserve some quantity along with its dipole moment. Shattering exists at any level of locality, but the dynamics are more shattered when locality is stronger. Other examples have been identified that only rely on an ordinary (monopole) symmetry, but the shattering completely breaks down when the dynamics act on a sufficiently large geometric region. This motivates two areas of study: multipole symmetries as a new class of exotic symmetries and finding examples of shattering that are robust to symmetry-breaking or locality-breaking perturbations.

Multipole symmetries have found many other applications in the literature, particular in connection to fracton models. In addition, they have been found to display interesting hydrodynamic signatures even beyond the effects of Hilbert-space shattering.

\section{Outline of Dissertation}

In what remains of this thesis, we will explore a small sample of exotic symmetries and how they lead to interesting quantum dynamics.

In Chapter~\ref{chp:WalkerWang}, we build a quantum memory inspired by Ref.~\cite{RobertsBartlett2020}. This chapter generalizes the  construction in three important ways: First, we confirm the prediction of Ref.~\cite{RobertsBartlett2020} that their construction works for any so-called Walker-Wang model. Then, we show that the symmetry only needs to be enforced within some region of the model rather than the entire bulk, Finally, we construct a model consisting of a trivial paramagnet that is nevertheless self-correcting with a judicious choice of symmetry.

In Chapter~\ref{chp:boundary}, we continue to expand on the work of the previous chapter by constructing a symmetry-protected quantum memory where the symmetry is only enforced on the 2d boundary. The new ingredient is a topologically ordered bulk, which possesses an emergent symmetry. Thus, a symmetry exists throughout the entire model but only needs to be enforced on the boundary. 

In Chapter~\ref{chp:single-shot}, we use the results of Ch.~\ref{chp:WalkerWang} to build a single-shot quantum memory. The advantages of this construction are the physical interpretability of the construction itself and of the resulting single-shot code. We study the physical properties of the code and interpret it as a many-body quantum system with a nontrivial phase diagram.

In Chapter~\ref{chp:superfluids}, we change gears to discuss multipole symmetries. Whereas the original motivation for these symmetries was the Hilbert-space shattering that occurs at early times, here we instead study the quantum dynamics at late times using the formalism of fluctuating hydrodynamics. We look at various hydrodynamic phases corresponding to different patterns of spontaneous symmetry breaking and how they appear in hydrodynamic dispersion relations. 

In Chapter~\ref{chp:loops}, we apply higher-form symmetries to the concept of Hilbert-space shattering in order to come up with models of topologically robust shattering. The shattering is robust in the sense that any dynamics compatible with the symmetry display shattering as long as they act on a finite number of spins. This is new because most existing models only display shattering for dynamics up to some finite size. We explore connections between our new model and existing models, and conjecture some other models that might display a stronger form of shattering.

Finally, in Chapter~\ref{chp:conclusion}, we wrap up the discussions of this thesis with some connections between the chapters, some open questions, and some directions for future research.