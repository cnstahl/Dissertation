% !TEX root = ../thesis.tex

\chapter{Conclusions}
\label{chp:conclusion}

\begin{quotation}
	 ``Begin at the beginning," the King said gravely, ``and go on till you come to the end; then stop."~[\hyperlink{cite.\therefsection @Carroll2002Alice}{Carroll}]
\end{quotation}

The overarching theme of the past chapters of this thesis has been finding interesting quantum dynamics via exotic symmetries. We explored new ways to build symmetry-protected self-correcting quantum memories using higher-form symmetries. Inspired by those results, we constructed new presentations of single-shot codes with familiar physical roots. We studied the hydrodynamics of fractonic systems in the weak-shattering regime, and we constructed new models of Hilbert-space shattering with topological robustness.

In many of the cases, we can interpret our results as information-storing many-body systems.
The interesting cases of many-body memories are when the logical information consists of qubits, as in a quantum memory, or consists of many bits, as in Hilbert-space shattering. It would be interesting to try to combine these into a quantum memory that stores many logical qubits. Models of 1d quantum shattering do exist in the literature~\cite{Moudgalya2022Commutant}, but without any sense of robustness. It remains to be seen whether it is possible to promote those 1d models to robust 2d models in the same way that Chapter~\ref{chp:loops} promotes the pair-flip model to the robust quad-flip model.

An obvious gap to try to close is to find Hilbert-space shattering that is robust to infinite times. In Chapter~\ref{chp:loops} we argued that the shattering should be robust out to a time exponentially long in the gap, but not beyond. Since the higher-form symmetry is continuous, it is likely that there should be some instanton-like effects that kick in at late time to melt the shattering. Thus, a similar model with only discrete higher-form symmetries would be likely to be more robust.

Similar models should also have some nice properties at non-zero temperature. Specifically, if it really is possible for shattering to persist to all times, then analysis similar to the suggestion in Sec.~\ref{sec:3D} should result in a finite energy barrier. This could lead to a model that is not ergodic even when coupled to a thermal bath, which would be new.

The single-shot memory of Chapter~\ref{chp:single-shot} descends from the symmetry-protected quantum codes of Ref.~\cite{RobertsBartlett2020} in an obvious way: the stabilizer group of the code is made up of the symmetry operators. Thus, it should be possible to construct single-shot quantum codes from any Walker-Wang model, as mentioned in Sec~\ref{sec:questions}. Is it also possible to build a single-shot code from the symmetry-protected memory in Chapter~\ref{chp:boundary}? If such a code exists, would it only have noncommuting checks on the boundary?

Another promising avenue is to explore local active error correction. The single-shot memories described in Chapter~\ref{chp:single-shot} rely on active error correction via local measurements, but also require nonlocal (in space) classical processing of those measurement outcomes. Fully local active error correction instead only uses local processing. Intriguingly, Ref.~\cite{Harrington2004Thesis} suggests that 2d\footnote{This is not a typo.} local active quantum error correction should be possible, based on the existence of 1d local active classical error correction~\cite{Gacs1983Reliable, Gacs2001Reliable, Gray2001Readers}. These 1d classical models are so complicated that very little is known about them, and they deserve further study. Even if Harrington's proposed 2d quantum memory is too complicated to be useful, it leaves open the possibility of a simpler 3d local active quantum memory. Whether exotic symmetries might be useful in describing such a system remains to be seen, but the existing links between symmetries and dynamics suggest this should be the case.